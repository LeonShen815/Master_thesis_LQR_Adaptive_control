
% This LaTeX was auto-generated from MATLAB code.
% To make changes, update the MATLAB code and republish this document.

\documentclass{article}
\usepackage{graphicx}
\usepackage{color}

\sloppy
\definecolor{lightgray}{gray}{0.5}
\setlength{\parindent}{0pt}

\begin{document}

    
    
\subsection*{Contents}

\begin{itemize}
\setlength{\itemsep}{-1ex}
   \item Linear And Nonlinear Tractor Trailer Controller ()
   \item Mass Matrix \%\%\%\%\%\%\%\%\%\%\%\%\%\%\%\%\%\%\%\%\%\%\%\%\%\%\%\%\%\%\%\%\%\%\%\%\%\%\%\%\%\%\%\%\%\%\%\%\%\%\%\%\%\%\%\%\%\%\%\%
   \item Stiffness K Matrix
   \item M inverse Times K \%\%\%\%\%\%\%\%\%\%\%\%\%\%\%\%\%\%\%\%\%\%\%\%\%\%\%\%\%\%\%\%\%\%\%\%\%\%\%\%\%\%\%\%\%\%\%\%\%\%\%\%\%\%
   \item delta and M inverse times delta \%\%\%\%\%\%\%\%\%\%\%\%\%\%\%\%\%\%\%\%\%\%\%\%\%\%\%\%\%\%\%\%\%\%\%\%\%\%\%\%
   \item Error Equations \%\%\%\%\%\%\%\%\%\%\%\%\%\%\%\%\%\%\%\%\%\%\%\%\%\%\%\%\%\%\%\%\%\%\%\%\%\%\%\%\%\%\%\%\%\%\%\%\%\%\%\%\%\%\%\%
   \item Caluclate indivual Matrix Parts
   \item Error States augmentation
\end{itemize}
\begin{verbatim}
function [A,B1,B2,D] = articulation1(m1,m2,a2,I1,I2,C,C3,Cs1,h1,l2,u,Cq1,C1,a1,l1)
\end{verbatim}


\subsection*{Linear And Nonlinear Tractor Trailer Controller ()}

\begin{par}
Simulates a tractor-trailer model
\end{par} \vspace{1em}
\begin{par}
\textbf{Author:} Kevin Gasik
\end{par} \vspace{1em}
\begin{par}
\textbf{Date:}   3/04/2019
\end{par} \vspace{1em}
\begin{par}
\textbf{Required Files:} (click for documentation)
\end{par} \vspace{1em}
\begin{par}
\textbf{See Also:}
\end{par} \vspace{1em}
\begin{par}
*
\end{par} \vspace{1em}
\begin{par}
\textbf{Code Repository:} Visit for source files
\end{par} \vspace{1em}
\begin{par}
\textbf{Detailed Description:} this function outputs a mass matrix vector to be used to control for an adaptive controller. This function takes 14 arguments
\end{par} \vspace{1em}
\begin{par}
\textbf{Copyright (C) 2019 Kevin Gasik} This file is not to be used or distributed by anyone without explicit consent from the author.This file serves as an example of how to approach and simulate these problems and should not be used to implement any controllers on any vehicle. By using the program shown below you agree that if anyone is harmed Kevin Gasik and California Polytechnic University is not liable.
\end{par} \vspace{1em}
\begin{verbatim}
b1 = l1-a1;
\end{verbatim}


\subsection*{Mass Matrix \%\%\%\%\%\%\%\%\%\%\%\%\%\%\%\%\%\%\%\%\%\%\%\%\%\%\%\%\%\%\%\%\%\%\%\%\%\%\%\%\%\%\%\%\%\%\%\%\%\%\%\%\%\%\%\%\%\%\%\%}

\begin{verbatim}
%this is the mass matrix associated with the lagragian derivation for
%The single articulated vehcile, we make sure to get this 3x3 matrix before
%making a 4x4 matrix
M = [(m1+m2), -m2*(h1+a2), -m2*a2;
    -m2*h1, (I1+m2*h1*(h1+a2)), m2*h1*a2;
    -m2*a2, (I2+m2*a2*(h1+a2)), I2+m2*a2^2];
%this is the mass matrix associated with the lagragian derivation for
%The single articulated vehcile
mass_matrix = [M(1,1), M(1,2),M(1,3), 0;
               M(2,1), M(2,2),M(2,3), 0;
               M(3,1), M(3,2),M(3,3), 0;
               0,      0,     0,      1];

M_inv = inv(mass_matrix);
\end{verbatim}


\subsection*{Stiffness K Matrix}

\begin{verbatim}
%this is the stiffness matrix K that is associated with the tractor-trailer
%model
K = [(C+C3),     (Cs1-C3*(h1+l2)+(m1+m2)*u^2),    (-C3*l2),    (-C3*u);
     (Cs1-C3*h1),(Cq1^2+C3*h1*(h1+l2)-m2*h1*u^2),  C3*h1*l2,    C3*h1*u;
     (-C3*l2),   (C3*l2*(h1+l2)-m2*a2*u^2),      (C3*l2^2),    (C3*l2*u);
     0,           0,                              -u,            0];
\end{verbatim}


\subsection*{M inverse Times K \%\%\%\%\%\%\%\%\%\%\%\%\%\%\%\%\%\%\%\%\%\%\%\%\%\%\%\%\%\%\%\%\%\%\%\%\%\%\%\%\%\%\%\%\%\%\%\%\%\%\%\%\%\%}

\begin{verbatim}
 MinvK = -1*M_inv*K/u;
\end{verbatim}


\subsection*{delta and M inverse times delta \%\%\%\%\%\%\%\%\%\%\%\%\%\%\%\%\%\%\%\%\%\%\%\%\%\%\%\%\%\%\%\%\%\%\%\%\%\%\%\%}

\begin{verbatim}
delta = [C1;a1*C1;0;0];
MinvDelta = M_inv*delta;
\end{verbatim}


\subsection*{Error Equations \%\%\%\%\%\%\%\%\%\%\%\%\%\%\%\%\%\%\%\%\%\%\%\%\%\%\%\%\%\%\%\%\%\%\%\%\%\%\%\%\%\%\%\%\%\%\%\%\%\%\%\%\%\%\%\%}

\begin{par}
we create the the error matrices as detailed in the thesis the entire process can be seen by kevin gasik's thesis on adaptive control to derive the error states look at chapter 3 and chapter 4.
\end{par} \vspace{1em}
\begin{verbatim}
eddot_A = [1,   0,          0,      0;
           0,   1,          0,      0;
           1,   0,          -a2,     0;
           0,   1,          1,      0];

eddot_B = [0,   u,          0,      0;
           0,   0,          0,      0;
           0,   u,          u,      0;
           0,   0,          0,      0];


Q3 = [-u,0,-u,0]'; % desired yaw rate matrix
%p1 p2 and p3 reprsents the state vector equal to components of the defined
%error equations
p1 = [1,        0,              0,      0;
      0,        1,              0,      0;
      1/a2,     0,           -1/a2,     0;
      0,        0,              0,      0];

p2 = [0,        -u,      0,      0;
      0,         0,      0,      0;
      0,     -u/a2,      0,   u/a2;
      0,        -1,      0,      1]; %% phi = -e2 + e4

p3 = [0;1;0;0]; % \dot \psi desired
\end{verbatim}


\subsection*{Caluclate indivual Matrix Parts}

\begin{par}
we break down this process to make it easier to calculate in the future to add together
\end{par} \vspace{1em}
\begin{verbatim}
Q1 = eddot_A*MinvK;
Q2 = eddot_A*MinvDelta; % delta matrix



C_matrix= Q1 + eddot_B; % matrix where x dot is are the states

X = C_matrix*p1;
Y = C_matrix*p2;
\end{verbatim}


\subsection*{Error States augmentation}

\begin{par}
here we add the components and aguement the error equations to include the lower order terms as well as the higher order terms
\end{par} \vspace{1em}
\begin{verbatim}
E_matrix = zeros(8,8);
E_matrix(1,:) = [0,1,0,0,0,0,0,0];
E_matrix(3,:) = [0,0,0,1,0,0,0,0];
E_matrix(5,:) = [0,0,0,0,0,1,0,0];
E_matrix(7,:) = [0,0,0,0,0,0,0,1];
E_matrix(2,:) = [X(1,1),Y(1,1),X(1,2),Y(1,2),X(1,3),Y(1,3),X(1,4),Y(1,4)];
E_matrix(4,:) = [X(2,1),Y(2,1),X(2,2),Y(2,2),X(2,3),Y(2,3),X(2,4),Y(2,4)];
E_matrix(6,:) = [X(3,1),Y(3,1),X(3,2),Y(3,2),X(3,3),Y(3,3),X(3,4),Y(3,4)];
E_matrix(8,:) = [X(4,1),Y(4,1),X(4,2),Y(4,2),X(4,3),Y(4,3),X(4,4),Y(4,4)];

%Outputs
A = E_matrix;

% delta output
B1 = Q2;
B1 = [0,B1(1),0,B1(2),0,B1(3),0,B1(4)]';

%yaw desired output
B2 = Q3 + C_matrix*p3;
B2 = [0,B2(1),0,B2(2),0,B2(3),0,B2(4)]';

D = 0;
\end{verbatim}
\begin{verbatim}
end
\end{verbatim}



\end{document}
    
