
% This LaTeX was auto-generated from MATLAB code.
% To make changes, update the MATLAB code and republish this document.

\documentclass{article}
\usepackage{graphicx}
\usepackage{color}

\sloppy
\definecolor{lightgray}{gray}{0.5}
\setlength{\parindent}{0pt}

\begin{document}

    
    
\subsection*{Contents}

\begin{itemize}
\setlength{\itemsep}{-1ex}
   \item Linear And Nonlinear Tractor Trailer Controller (Trailer Constants)
\end{itemize}
\begin{verbatim}
function [g,m2,a2,l2,b2,e1,k2,I2,Fz3] = trailer_constants(g,m2)
\end{verbatim}


\subsection*{Linear And Nonlinear Tractor Trailer Controller (Trailer Constants)}

\begin{par}
Assigns the trailer constants for the main script to use.
\end{par} \vspace{1em}
\begin{par}
\textbf{Author:} Kevin Gasik
\end{par} \vspace{1em}
\begin{par}
\textbf{Date:}   3/04/2019
\end{par} \vspace{1em}
\begin{par}
\textbf{See Also:}
\end{par} \vspace{1em}
\begin{par}
*
\end{par} \vspace{1em}
\begin{par}
\textbf{Code Repository:} Visit for source files
\end{par} \vspace{1em}
\begin{par}
\textbf{Detailed Description:} This function takes two inputs g-gravity and m2-mass of the trailer to then assign the rest of the trailer constants and output them to the main script.
\end{par} \vspace{1em}
\begin{par}
\textbf{Copyright (C) 2019 Kevin Gasik} This file is not to be used or distributed by anyone without explicit consent from the author.This file serves as an example of how to approach and simulate these problems and should not be used to implement any controllers on any vehicle. By using the program shown below you agree that if anyone is harmed Kevin Gasik and California Polytechnic University is not liable.
\end{par} \vspace{1em}
\begin{verbatim}
g = g;
%%Define trailer constants
m2 = m2; % trailer mass (kg)
a2 = 4.98; % distance from front tire of trailer to trailer center of mass (m)
l2 = 8.13; % distance from front tire to back tire of trailer (m)
b2 = l2 - a2;
e1 = -0.68; % distance rear axel towing point (m)
k2 = 4.05; % raidus of gyration for trailer (m)
I2 = m2*k2^2; % moment of inertia for trailer
Fz3 = a2/l2*m2*g; % Force on the rear wheel of trailer when not moving
\end{verbatim}
\begin{verbatim}
end
\end{verbatim}



\end{document}
    
